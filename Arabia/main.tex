\documentclass{beamer}

\usepackage[alf]{abntex2cite}
\usepackage{color}
\usepackage{graphicx}
\usepackage{enumitem}
\usepackage{fontspec}
\usepackage{polyglossia}

\setdefaultlanguage{brazil}
\setotherlanguage[calendar=gregorian,numerals=maghrib]{arabic}
\newfontfamily\arabicfont[Script=Arabic,Scale=1]{Amiri}
\newfontfamily\arabicfontsf[Script=Arabic,Scale=1]{Amiri}

% configures the presentations to use \section and \subsection as tableofcontents
% in a way that it automatically creates lovely slides all over the place
% correctly filling the CPEL approach needed by universities.

\DeclareRobustCommand{\sectioncontents}[1]{
	\AtBeginSection[] {
		\begin{frame}
			\frametitle{#1}
			\tableofcontents[currentsection]
		\end{frame}
	}
}

\DeclareRobustCommand{\subsectioncontents}[1]{
	\AtBeginSubsection[]{
		\begin{frame}
			\frametitle{#1}
			\tableofcontents[currentsection,currentsubsection]
		\end{frame}
	}
}

\title{ Arábia nos tempos de glória \\( ou o testamento da ira )}
\author{ Tomaz {Canabrava} \and Carla {Sobrenome}}
\institute{ IFSP - HTO \\ Instituto federal de São Paulo, Campus Hortolândia}
\date{\today}

\sectioncontents{Conteudo}
\subsectioncontents{Conteudo}

\begin{document}

\begin{frame}
	\begin{center}
		\textarabic{﷽\\أبدأ مداخلتي\\ لا إله إلا الله محمد رسول الله}
	\end{center}
\end{frame}

\begin{frame}
	\begin{center}
		Em nome de Allah, o compassivo, o Todo misericordioso\\
		Eu inicio minha apresentação\\
		Pois não há Deus senão Allah e Mohhamed é seu profeta
	\end{center}
\end{frame}

\begin{frame}
 \titlepage
\end{frame}

\section{A Era de Ouro}
\subsection{Influência Cultural}

\begin{frame} \frametitle{ Assimilação Cultural \hfill \textarabic{﷽}}
	\begin{itemize}
		\item Tradução Literária
		\item Centros de Estudos
		\item Hospitais
		\item Bibliotecas
		\end{itemize}
\end{frame}

\begin{frame} \frametitle{ Tradução Literária \hfill \textarabic{﷽}}
	Tradução e importação de estudiosos de diversas civilizações
	\begin{columns}
		\column{0.5\textwidth}
		\begin{itemize}
			\item<1-> Grécia
			\item<2-> Roma
			\item<3-> Persia
			\item<4-> India
			\item<5-> China
			\item<6-> Egito
			\item<7-> Fenicia
		\end{itemize}
		\column{0.5\textwidth}
		\begin{figure}
			\includegraphics<1>[width=\textwidth,height=6cm,keepaspectratio]{images/siege-of-thesalonica}
			\includegraphics<2>[width=\textwidth,height=6cm,keepaspectratio]{images/siege-of-constantinopla}
			\includegraphics<3>[width=\textwidth,height=6cm,keepaspectratio]{images/sack-of-persia}
			\includegraphics<4>[width=\textwidth,height=6cm,keepaspectratio]{images/sack-of-india}
			\includegraphics<5>[width=\textwidth,height=6cm,keepaspectratio]{images/abasid-caliphate}
			\includegraphics<6>[width=\textwidth,height=6cm,keepaspectratio]{images/fatmid-caliphate}
			\includegraphics<7>[width=\textwidth,height=6cm,keepaspectratio]{images/battle-of-yamouk}
			\caption{
				\only<1>{Saque da Thessalonica}
				\only<2>{Saque de Constantinopla}
				\only<3>{Saque do Império Persa}
				\only<4>{Saque da India}
				\only<5>{Não Saque da China}
				\only<6>{Saque do Egito}
				\only<7>{Saque do Império Fenicio}
			}
\end{figure}
	\end{columns}
\end{frame}

\subsection{ Influência Religiosa }

\begin{frame} \frametitle{ Influência Religiosa \hfill \textarabic{﷽}}
	\begin{itemize}
		\item Quoran incentiva os estudos \pause
		\item Leia, em nome de Allah, seu criador. (Quran 96:1-2) \pause
		\item Ó, Allah! permita-me conhecer. (Quran 20:114) \pause
		\item Buscar o conhecimento é obrigatório para cada muçulmano. (Mohammed )
	\end{itemize}
\end{frame}

\subsection{ Influência Governamental }

\subsection{ Influência Tecnológica }
\section{Jihad}
\subsection{ Cruzadas }
\section{Atualidade}

\end{document}
